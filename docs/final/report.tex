% Classe do documento e parâmetros gerais.
\documentclass[a4paper,openright,twoside,11pt]{report}

% Packages utilizadas e respetivos parâmetros.
\usepackage[utf8]{inputenc}
\usepackage[english]{babel}

\usepackage{lipsum} % gerador de texto
\usepackage{graphicx}
\usepackage{url}
\usepackage[Algoritmo]{algorithm}
\usepackage{algorithmicx}
\usepackage{algpseudocode}
\usepackage[printonlyused]{acronym}
\usepackage{listings}
\usepackage{hyperref}
\usepackage{tikz}
\usetikzlibrary{arrows.meta}
\renewcommand{\algorithmicrequire}{\textbf{Dados: }}
\renewcommand{\algorithmicensure}{\textbf{Resultado: }}

% Definições das dimensões das páginas
\setlength{\textheight}{24.00cm}
\setlength{\textwidth}{15.50cm}
\setlength{\topmargin}{0.35cm}
\setlength{\headheight}{0cm}
\setlength{\headsep}{0cm}
\setlength{\oddsidemargin}{0.25cm}
\setlength{\evensidemargin}{0.25cm}

%\renewcommand{\baselinestretch}{1}

% Página inicial (capa)
\title{
    \vspace{-50mm}
    \begin{minipage}[l]{\textwidth}
        \hspace{-20mm}\resizebox{75mm}{!}{\includegraphics{./figures/logoISELnew2}}\\
    \end{minipage}\\[20mm]
    {\Huge \textbf{Krometheus}}\\[3mm]
    {\Large A Prometheus Kotlin client library}\\[10mm]
}

% Nome dos autores (um por linha)
\author{
    \begin{tabular}{ll}
        & Mário Carvalho \\
        & Rafael Nicolau \\[50mm]
    \end{tabular}}

\date{
    \begin{tabular}{ll}
    {Supervisor:}
        & José Simão \\
    \end{tabular}\\[10mm]
% Deixar o indicador respetivo em função da versão do relatório.
    Final report written for Project and Seminary\\
    BSc in Computer Science and Computer Engineering\\[20mm]
    June 2025}


\begin{document}
    \thispagestyle{empty}
    \maketitle

    \baselineskip 18pt % line spacing: 12pt for single, 18pt for 1 1/2, and 24pt for double spacing

    \newpage
    \thispagestyle{empty}
% Fim da contracapa

% Página com identificação completa (número e nome) e assinaturas do(s) estudante(s) e do(s) orientador(es)
    \cleardoublepage
    \setcounter{page}{1}
    \begin{center}
        \textsc{\LARGE Instituto Superior de Engenharia de Lisboa}\\[50mm]

        {\Huge \textbf{Krometheus}}\\[3mm]
        {\Large A Prometheus Kotlin client library}\\[10mm]
        \begin{tabular}{rl}
            50561 & Mário Rijo Antunes Carvalho\\[10mm]
            & \rule{75mm}{0.5pt}\\[5mm]
            50546 & Rafael Sebastião Matias Nicolau\\[10mm]
            & \rule{75mm}{0.5pt}              \\
        \end{tabular}\\[10mm]

        \begin{tabular}{rl}
            Supervisor: & José Manuel de Campos Lages Garcia Simão\\[10mm]
            & \rule{75mm}{0.5pt}\\[5mm]
        \end{tabular}\\[10mm]

        Final report written for Project and Seminary\\
        BSc in Computer Science and Computer Engineering\\[20mm]
        June 2025\\
    \end{center}

% Página de resumo em ingles
    \cleardoublepage
    \chapter*{Abstract}
    The growing complexity of modern software systems, driven by the adoption of distributed architectures,
    microservices, and asynchronous execution—has made observability a critical aspect of system design.
    Among its key components, metrics-based monitoring enables developers and operators to track the internal state and performance of applications and infrastructure.
    Prometheus has become a widely used standard for collecting and querying metrics; however, its official client libraries primarily target languages such as Go, Python, and Java.
    These implementations do not offer idiomatic support for Kotlin, its coroutine-based concurrency model, or
    Kotlin-specific frameworks like Ktor.\\

    This project presents a Prometheus client library implemented in Kotlin, designed with coroutine safety, expressiveness, and alignment with Kotlin's language features in mind.
    The library supports core Prometheus metric types and provides a straightforward and testable interface for instrumenting Kotlin applications.
    This report details the motivation for the project, its design and architecture, the implementation process, and an evaluation of the resulting system.
    The library fills an existing gap in the Kotlin ecosystem by offering a native and idiomatic solution for
    integrating metrics-based observability.\\

    {\textbf{Keywords}: Observability, Prometheus, Metrics, Monitoring.}

%% Página de agradecimentos
    \cleardoublepage
    \chapter*{Acknowledgments}
    \lipsum[1]
    \vspace{2em}
    \begin{flushright}
        Rafael Nicolau
    \end{flushright}

    \lipsum[1]
    \vspace{2em}
    \begin{flushright}
        Mário Carvalho
    \end{flushright}

% Geração do índice de conteúdos
    \cleardoublepage
    \tableofcontents \cleardoublepage

% Geração do índice de figuras e de tabelas
    \listoffigures \cleardoublepage
    \listoftables \cleardoublepage
    \chapter*{List of Abbreviations}
\addcontentsline{toc}{chapter}{List of Abbreviations}

\begin{acronym}[TDMA]  % widest acronym to set the width for the list
    \acro{ARPANET}{Advanced Research Projects Agency Network}
    \acro{ENIAC}{Electronic Numerical Integrator and Computer}
    \acro{WWW}{World Wide Web}
    \acro{TCP/IP}{Transmission Control Protocol/Internet Protocol}
    \acro{FTP}{File Transfer Protocol}
    \acro{SSH}{Secure Shell}
    \acro{SOA}{Service-oriented Architecture}
    \acro{SaaS}{Software as a Service}
    \acro{EC2}{Elastic Compute Cloud}
    \acro{VM}{Virtual Machine}
    \acro{SLA}{Service Level Agreement}
    \acro{API}{Application Programming Interface}
    \acro{KMP}{Kotlin Multiplatform}
    \acro{MTTR}{Mean Time To Repair}
    \acro{CNCF}{Cloud Native Computing Foundation}
    \acro{CPU}{Central Processing Unit}
    \acro{JVM}{Java Virtual Machine}
    \acro{DSL}{Domain-Specific Language}
\end{acronym}
 \cleardoublepage

% Chapter 1
    %
% Chapter 1
%
\chapter{Introduction} \label{ch:introduction}


\section{Context and Motivation}\label{sec:context-and-motivation}
Over the past few decades, we have been witnessing remarkable advancements on our software systems
and their architecture.\\

The first computers, mainframe computers, Harvard Mark I and the \ac{ENIAC} showed up in the 1930s-1940s and where
developed for military and research purposes.
These were large, powerful hardware machines that took up to an entire room.
The software architecture for these applications was monolithic, meaning that they were simple, self-contained and
independent of other applications~\cite{orkes_software_architecture_evolution, wikipedia_monolithic}.\\

Networks connect and facilitate communication between computers--mainframe to terminal, mainframe to mainframe and
later client to server.
The development of network technology from 1958 onwards enabled mainframes to be connected
electronically, transforming them from isolated machines to multi-user computers that were connected to multiple
terminals.
\ac{ARPANET} was the first public, wide-area computer network, going live in 1969.
It communicated using
packet switching, which went on to serve as the foundation for modern-day Internet as we know it.
Network technology popularized the client-server structure in the 80s, where an application is divided into a
server and a client communicate over a network.
This is a very familiar structure to us today: a client, typically a
desktop computer, remotely makes a request to a server, which returns a response.
This way we can partition tasks
and workloads, the server will work on the processing and retrieval of a piece of data, and the client will in turn
present it~\cite{orkes_software_architecture_evolution, wikipedia_client_server_model}.\\

1983 marked the year of the Internet, a global system of computer networks that used the \ac{TCP/IP} protocol to
facilitate communication between devices and applications.
This was the backbone for \ac{FTP} programs, \ac{SSH} systems, and of course, the World Wide Web.
The invention of the web and its latent possibilities soon kicked off the next wave of application development.
Instead of building a dedicated client for your client for your application, you could simply build a website to be
hosted on the web~\cite{orkes_software_architecture_evolution}.\\

As application development grew, a monolithic codebase became more unwieldy to manage, and it became clear that
capabilities and data house in a system could be reused.
To address this pain point, modularization became a topic of discussion, and in the 90s the first N-tiered
applications showed up, for the first time we saw the server being split into two tiers: the application server and
the database, the application held all the application and business logic, while the database server, stored the
data recorded, which reduced latency at high processing volumes.
Around the same time, \ac{SOA} emerged as an architectural pattern, and we saw applications split into multiple
services, each design to perform a specific job which could be things like processing payments or verifying user
identities, this had benefits such as an increased scalability as single services could be scaled up or down
depending on demand without having to make changes to the entire system
~\cite{orkes_software_architecture_evolution, oracle_soa, port_soa}.\\

\ac{SOA} set the stage for the move from traditional desktop applications to a new mode of software applications - \ac{SaaS},
but it was the invention of virtual machines and cloud computing that further spurred the explosion of \ac{SaaS}
products in the coming decades.
Machine virtualization exists since the 1960s, but this technology only came into
mainstream use in the 2000s.
It at this period that Amazon, followed by companies like Google, Microsoft and Oracle identified the lucrative that
virtualization offered: managed cloud computing services.
With cloud computing services like Amazon \ac{EC2}, companies could rent \ac{VM}s for processing power and scale as
needed~\cite{orkes_software_architecture_evolution}.\\

The 2010s were the culmination of multiple trends towards distributed computing.
Fueled by the need for third-party access to their services, the first commercial \ac{API}s were lauched in the
2000 by Salesforce and eBay, to enable their costumers and partners to integrate features onto their own sites and
applications.
From Google Maps and Stripe to Twilio and OpenAI, the \ac{API} economy has ballooned since, powering integrated
features across the web.
In the same vein, microservices took off when scaling companies like Netflix and Amazon needed to speed up and
streamline the development cycle, which was slowed by a monolithic architecture.
By splitting up an application into individual microservices, each with its own database, teams could independently update and deploy them, leading to
faster releases and improvements.\cite{orkes_software_architecture_evolution}
The growth in the adoption of this architectural style was heavily supported by the emergence of containers.\\

As we move into the present decade, software systems have become increasingly complex and distributed.
Modern applications often consist of hundreds or thousands of microservices, deployed across hybrid and multi-cloud environments, communicating asynchronously via APIs, message queues, and event streams.
Containers and orchestration platforms like Kubernetes have become the de facto standard for managing these
distributed components at scale\cite{devtron_kubernetes}.
While this architectural evolution brings immense benefits—such as scalability, resilience, and faster development cycles—it also introduces significant operational challenges.
The large number of independent moving parts, dynamic scaling, and network dependencies make it difficult to predict
and diagnose failures or performance degradation using traditional reactive methods.\\

Consequently, couple with more stringent \ac{SLA}s than ever, for multiple reasons, organizations now emphasize
observability, the
ability to
measure a
system's
current
state based
on its
external outputs, recorded as metrics, logs, and traces\cite{dynatrace_observability}.
Observability enables a proactive approach to maintaining a systems reliability and performance standards, allowing
teams to detect
anomalies
early,
understand complex interactions, and respond swiftly before issues impact end users.\\


In this context, metrics-based monitoring, has become essential for maintaining the health of modern distributed systems.
This project aims to fill a notable gap in the Kotlin ecosystem by providing a robust and idiomatic Prometheus client library tailored specifically for Kotlin applications.
Unlike existing libraries that primarily target other JVM languages or lack native support for Kotlin’s coroutine model, this library offers seamless integration and out-of-the-box support for popular Kotlin-only frameworks such as Ktor.
By delivering a solution designed with Kotlin’s language features and concurrency paradigms in mind, this project enables developers to instrument their applications effectively and embrace modern metrics-based monitoring with minimal friction.

\section{Objectives}\label{sec:objectives}

The primary objective of this project is to develop a Kotlin-native Prometheus client library that addresses existing limitations in the Kotlin ecosystem by providing a robust, idiomatic, and efficient solution for metrics instrumentation and collection.

Specifically, the project aims to:

\begin{itemize}
    \item Design and implement a client library that fully leverages Kotlin’s coroutine-based concurrency model to enable asynchronous and efficient metric collection.
    \item Provide seamless integration with popular Kotlin frameworks such as Ktor, ensuring ease of use and compatibility within typical Kotlin application environments.
    \item Ensure the library is designed with readability and idiomatic Kotlin style in mind, making it intuitive and enjoyable to use.
    \item Develop an expressive and idiomatic Domain-Specific Language (DSL) for defining and managing metrics, reducing boilerplate code and improving developer productivity.
    \item Optimize the library for performance and resource efficiency, surpassing existing Java-based Prometheus
    clients when used in Kotlin applications where possible.
    \item Support \ac{KMP}, enabling broad platform compatibility and facilitating code reuse across JVM, Android, and
    other supported targets.
\end{itemize}

By achieving these objectives, the project intends to fill a critical gap in Kotlin observability tooling and empower developers to instrument their applications with minimal effort while adhering to modern Kotlin programming paradigms.

\section{Structure of the Report}\label{sec:structure-of-the-report}

This report is organized into eight main chapters, each focusing on a specific aspect of the project:

\begin{itemize}
    \item \textbf{Chapter 1: Introduction} — Presents the context and motivation behind the project, outlines the objectives, and provides an overview of the report’s structure.

    \item \textbf{Chapter 2: Background and Motivation} — Covers foundational concepts related to observability and Prometheus, reviews existing Prometheus client libraries, and highlights gaps within the Kotlin ecosystem that motivate this work.

    \item \textbf{Chapter 3: Problem Statement and Goals} — Defines the specific limitations identified in current tooling and states the precise goals and scope of the project.

    \item \textbf{Chapter 4: Design and Architecture} — Describes the high-level architectural decisions, focusing on concurrency models, coroutine usage, and the lifecycle of metric registries.

    \item \textbf{Chapter 5: Implementation Details} — Details the concrete implementation of the client library, including metric types, collector logic, and framework integrations such as with Ktor.

    \item \textbf{Chapter 6: Evaluation and Limitations} — Presents the functional validation, performance evaluation, and comparison with other clients, as well as known limitations of the project.

    \item \textbf{Chapter 7: Future Work} — Suggests possible extensions and improvements, including support for additional technologies, enhanced integration with Kotlin frameworks, and automatic JVM metrics collection.

    \item \textbf{Chapter 8: Conclusion} — Summarizes the contributions made by the project, reflects on lessons learned, and offers final remarks.
\end{itemize}

The report concludes with a list of references and an appendix containing supplementary material.


% Chapter 2
    %
% Chapter 2
%
\chapter{Background and Motivation} \label{ch:background}


\section{Observability Overview}\label{sec:observability-overview}
Observability refers to the ability to understand the internal state of a system by examining the data it generates~\cite{ibm_observability}.\\

\subsection{Observability vs Monitoring}\label{subsec:observability-vs-monitoring}
Not to be confused with monitoring, while both terms are sometimes used interchangeably they refer to distinct
processes and should be used in tandem to successfully maintain and manage the health and performance of modern day
software systems and their infrastructures~\cite{aws_observability_vs_monitoring}.\\

The key differences between both are that monitoring is mainly focused on the collection of data to identify
anomalous system effects, typically concerned with standalone systems and limited to the edges of the system, while
observability would is mainly focused on the investigation of the root cause of anomalous system effects, more often
than not concerned with multiple, disparate systems.
Monitoring tell us the \textit{when} and \textit{what}, and observability tells us the \textit{why} and \textit{how}.\\

\subsection{The pillars of observability}\label{subsec:the-pillars-of-observability}
There are three telemetry types that, together make up what is usually called,
\textbf{the three pillars of observability} - Logs, metrics and traces~\cite{ibm_observability}.\\

\begin{figure}[H]
    \centering
    \includegraphics[width=\linewidth, keepaspectratio]{./figures/observability_pillars}
    \caption{The three pillars of observability}
\end{figure}


\textbf{Logs} - Logs are time-stamped, complete and immutable records of specific events that occur within a system.
They offer a granular view of what happened within a system, helping with debugging or with understanding
events.
Examples include error messages, system events or transaction records.\\

\begin{figure}[h]
    \centering
    \begin{lstlisting}
2025-06-01T12:34:56Z INFO User admin123 logged in from IP 192.168.1.251
    \end{lstlisting}
    \caption{Example of a structured log message}
\end{figure}


\textbf{Metrics} - Metrics provide a broad view of the health of a system over time, they consist of quantitative
pieces
of data that
measures various aspects of system performance and resource utilization, allowing for trend analysis and forecasting.
For example, metrics can be used to measure how much memory a system is using for a given amount of time, or to
measure the latency experienced during a usage spike.\\
\begin{figure}[h]
    \centering
    \begin{lstlisting}
node_cpu_seconds_total{mode="user"} 12345.67
    \end{lstlisting}
    \caption{Example of a Prometheus-style metric}
\end{figure}


\textbf{Traces} - Traces are records that track the whole lifecycle of a request through all the components that
make up
the system.
Traces help in understanding the path and performance of requests, being key in identifying potential bottlenecks,
and with the diagnosis of latency issues.
One example of a trace would be, a record showing how a user request travels through different microservices.\\
\begin{figure}[h]
    \centering
    \begin{lstlisting}
12:00:01 - Service A received request
12:00:02 - Service A called Service B
12:00:03 - Service B responded
12:00:04 - Service A returned response
    \end{lstlisting}
    \caption{Example of a distributed trace log}
\end{figure}


\subsection{Benefits of Observability}\label{subsec:benefits-of-observability}

Full-stack observability can make a system easier to understand and monitor, easier and safer to update and easier 
to repair as it gives teams the ability to~\cite{ibm_observability}:

\begin{itemize}
    \item   \textbf{Minimize downtime and \ac{MTTR}}
    \item   \textbf{Automate remediation and self-healing application infrastructure}
    \item   \textbf{Scale automatically}
    \item   \textbf{Improve the user experience}
    \item   \textbf{Identify and resolve issues early in development}
    \item   \textbf{Discover and address "unknown unknowns"}
\end{itemize}
\section{Prometheus Overview}\label{sec:prometheus-overview}
Prometheus is an open-source systems monitoring and alerting toolkit originally built at SoundCloud in 2012.
Prometheus has, since then, become one of the most popular monitoring tools, seeing major adoption across both the
industry and the open-source community.
It is now a standalone open-source project maintained independently of any company, joining the \ac{CNCF} in 2016 as
the second hosted project after Kubernetes~\cite{what_is_prometheus}.\\

\subsection{Features}\label{subsec:features}

Prometheus main features are:

\begin{itemize}
    \item Organizes data as time series, each uniquely identified by a metric name and a set of key-value labels.
    \item PromQL, a powerful query language that takes advantage of this label-based dimensionality.
    \item Runs as a standalone server without depending on distributed storage, ensuring each node operates
    independently.
    \item Collects data over HTTP using a pull-based model, though pushing metrics is also possible using a gateway.
    \item Configuration of targets can be done statically or dynamically via service discovery.
    \item Data visualization is supported through various graphing tool and dashboard integrations.
\end{itemize}

\subsection{Design and Architecture}\label{subsec:design-and-architecture}

\begin{figure}[H]
    \centering
    \includegraphics[width=\linewidth, keepaspectratio]{./figures/prometheus_arch}
    \caption{Prometheus Architecture}
\end{figure}


The prometheus ecosystem is made up of various components, some of which are optional:

\begin{itemize}
    \item \textbf{Prometheus Server}: scrapes and store data on the \ac{TSDB}.
    \item \textbf{Client Libraries}: Instrument application code, exposing internal metrics in a
    Prometheus-compatible format.
    \item \textbf{Push Gateway}: Enables push-based metric collection for short-lived jobs or batch processes that
    can't be scraped directly.
    \item \textbf{Alertmanager}: Handles alerts sent by the main server, managing silencing, inhibition, grouping, and routing to external notification systems like email, Slack, or Discord.
    \item \textbf{Exporters}: Bridge components that expose metrics from third-party systems and allows for monitoring of
    systems that can't be instrumented directly.
    \item \textbf{Service Discovery}: Allows for automatic discovery of targets via integrations with Kubernetes,
    Amazon EC2, Docker Swarm, and many more.
    \item \textbf{Visualization tools}: Includes a basic UI and expression browser, with the possibility of using
    different data visualization libraries such as Grafana for richer dashboards.
\end{itemize}

Prometheus fundamentally stores all data as \textit{time series}: streams of timestamped
values belonging to the same metric and the same set of labeled dimensions~\cite{prometheus_data_model}.\\

\begin{figure}[H]
    \centering
    \includegraphics[width=0.9\linewidth, keepaspectratio]{./figures/samples_diagram}
    \caption{Prometheus Data Model}
\end{figure}


Time series are uniquely identified by their metric name and optional key-value pairs called labels.\\
The names of time series must comply with the following rules:

\begin{itemize}
    \item It should specify the general feature of a system that is measured.
    \item It may use any UTF-8 characters.
    \item It should match the regex \texttt{[a-zA-Z\_:][a-zA-Z0-9\_:]*} for the best experience and
    compatibility.
    Metric names outside of that set will require quoting.
\end{itemize}

Metric labels must comply with the following rules:

\begin{itemize}
    \item Both names and values may use any UTF-8 characters.
    \item If the name of a metric begins with two underscores, said metric must be reserved for internal
    Prometheus use.
    \item It should match the regex \texttt{[a-zA-Z\_:][a-zA-Z0-9\_:]*} for the best experience and
    compatibility.
    Metric names outside of that set will require quoting.
    \item Labels with an empty label value are considered equivalent to labels that do not exist.
\end{itemize}


Time series are identified using the following notation:
\begin{figure}[h]
    \centering
    \begin{lstlisting}
<metric name>{<labelN name>="<labelN value>", ...}
    \end{lstlisting}
    \caption{Example of a distributed trace log}
\end{figure}

Samples form the actual time series data, they are the most basic unit of data in Prometheus.
Each sample consists of a float64 or native histogram value, and a millisecond-precision timestamp.\\

\subsection{Metric Types}\label{subsec:metric-types}

Prometheus client libraries offer four core metric types, as of the time of writing of this report the Prometheus
server does not make use of the type information of these metrics and all data is flattened to untyped time series.
Only client libraries, to enable \ac{API}s tailored to the usage of specific types and the wire protocol make use of
type information of the metrics~\cite{prometheus_metric_types}.\\

\begin{figure}[H]
    \centering
    \includegraphics[width=\linewidth, keepaspectratio]{./figures/metric_types}
    \caption{Prometheus Metric Types}
\end{figure}


\textbf{Counter} - A counter represents a cumulative metric that a value whose value can only change upwards with
the ability of being reset to zero on restart.
An example use case would be counting the number of requests made to an \ac{API}.\\

\textbf{Gauge} - A gauge is a metric that represents a value that can change either upwards or downwards, typically
used for measured values like temperatures, disk utilization or, for example, the number of active connections
to a database at a given moment.

\textbf{Histogram} – A histogram samples observations (usually things like request durations or response sizes) and counts them in configurable buckets.
It provides a way to understand the distribution of values by showing how many observations fall into each bucket.
Histograms also automatically calculate the sum and count of all observed values, enabling the computation of averages and percentiles.

\textbf{Summary} – A summary also samples observations and provides a total count, sum, and configurable quantiles (percentiles) over a sliding time window.
Unlike histograms, summaries calculate quantiles on the client side, making them useful for tracking latency distributions or other metrics where precise quantiles are needed.
However, summaries are not aggregated across multiple instances as easily as histograms.


\section{Existing Prometheus Clients}\label{sec:existing-prometheus-clients}

Prometheus supports a variety of official client libraries that enable instrumenting applications to expose
metrics in a format compatible with the Prometheus server.
These official clients cover popular programming languages such as Go, Java, Python, Ruby, and Rust.
Additionally, the Prometheus community encourages developers to create and maintain client libraries for languages not yet officially supported, fostering a diverse ecosystem of instrumentation tools.

The official client libraries include:
\begin{itemize}
    \item \texttt{client\_golang} for Go
    \item \texttt{client\_java} for Java and \ac{JVM} languages
    \item \texttt{client\_python} for Python
    \item \texttt{client\_ruby} for Ruby
    \item \texttt{client\_rust} for Rust
\end{itemize}

Some notable unofficial (community-driven) client libraries include:
\begin{itemize}
    \item \texttt{prometheus-client-c} for C
    \item \texttt{prometheus-cpp} for C++
    \item \texttt{prometheus-net} for .NET/C\#
    \item \texttt{prom-client} for Node.js
    \item \texttt{prometheus\_client\_php} for PHP
\end{itemize}

This ongoing effort helps ensure Prometheus instrumentation can be integrated across a wide range of applications and environments.

\section{The Kotlin Ecosystem and Observability Gaps}\label{sec:the-kotlin-ecosystem-and-observability-gaps}

While the official Prometheus Java client library is widely used across the \ac{JVM} ecosystem, it is primarily
designed with Java’s language features and idioms in mind.
This design choice limits its ability to fully leverage Kotlin-specific features such as coroutines, extension functions, and null safety.
Consequently, using the Java client directly in Kotlin projects can lead to suboptimal performance and non-idiomatic code patterns.

For example, popular Kotlin frameworks like \texttt{Ktor} currently lack out-of-the-box support from the official Prometheus client.
This gap forces developers to write custom wrappers or adapters, increasing both development complexity and maintenance overhead.
Additionally, the Java client does not always seamlessly integrate with Kotlin’s coroutine-based concurrency model, which can hinder efficient metric instrumentation.

Recognizing these challenges, the Prometheus project encourages the development of language-specific client libraries.
It provides clear guidelines on best practices and pitfalls to avoid when creating such libraries, helping ensure consistency and quality across implementations.
Moreover, Prometheus maintains an active mailing list where developers can seek clarification and feedback directly from core Prometheus maintainers.
This open line of communication supports community-driven efforts to fill observability gaps in ecosystems like Kotlin, fostering better integration and more idiomatic instrumentation libraries.


\section{Identified Limitations}

While Prometheus has become the de facto standard for metrics-based observability in cloud-native systems, its official client libraries are largely written in imperative languages such as Go, Java, and Python. These implementations do not align well with Kotlin’s idiomatic programming model, particularly regarding coroutine support, functional DSLs, and thread-safety via structured concurrency.

The official Java Prometheus client, while usable in Kotlin projects, exposes a verbose and boilerplate-heavy API. Furthermore, it does not integrate naturally with Kotlin coroutines, leading to potential blocking behavior or unsafe metric updates when used improperly in suspendable contexts.

Additionally, Kotlin developers lack a native library that fully embraces Kotlin-specific features like extension functions, scope builders, and coroutine-friendly primitives for metrics definition and recording. This mismatch creates friction when building observability into modern reactive applications using frameworks such as Ktor or Spring WebFlux.

\section{Project Goals}

This project aims to develop a lightweight, idiomatic Prometheus client library written in Kotlin. The primary goals are:

\begin{itemize}
  \item \textbf{Kotlin-first design:} Create a metrics API that aligns with Kotlin’s language features, using builder functions, lambdas, and extension patterns to reduce boilerplate and improve readability.
  \item \textbf{Coroutine safety:} Ensure that metric operations can be safely invoked from within suspending functions and concurrent coroutines, without introducing race conditions or blocking behavior.
  \item \textbf{OpenMetrics compatibility:} Conform to the OpenMetrics exposition format, ensuring compatibility with Prometheus scraping, tooling, and external monitoring pipelines.
  \item \textbf{Core metric support:} Implement the most commonly used metric types—\texttt{counter}, \texttt{gauge}, \texttt{histogram}, and \texttt{summary}—with support for labeling, concurrency, and serialization.
  \item \textbf{Extensibility:} Provide a modular and extensible architecture, allowing future features such as HTTP exposition, JVM metrics integration, or PushGateway support to be added incrementally.
\end{itemize}

\section{Out of Scope}

To maintain focus and deliver a stable MVP, certain features and use cases are intentionally excluded from this project:

\begin{itemize}
  \item \textbf{PushGateway integration:} The library will support pull-based scraping only. Push-based metric transport (e.g., batch job metrics) is not addressed.
  \item \textbf{Advanced metric types:} Specialized types like \texttt{info}, \texttt{enum}, or \texttt{state\_set} will not be implemented in the initial version.
  \item \textbf{JVM internals instrumentation:} Metrics related to garbage collection, memory pools, or thread activity are considered outside the scope of this project.
\end{itemize}


% Chapter 3
    \chapter{Problem Statement and Goals} \label{ch:problemdescription}


\section{Identified Limitations}
\lipsum[1]


\section{Project Goals}
\lipsum[1]


\section{Out of Scope}
\lipsum[1]


% Chapter 4
    \chapter{Design and Architecture} \label{ch:architecture}

\section{Overview}\label{sec:overview}

The Prometheus Kotlin client library is architected to provide a comprehensive, \ac{KMP} solution for
defining, collecting, and exporting application metrics in a manner compliant with the Prometheus monitoring ecosystem.
Its design supports all core Prometheus metric types—namely, counters, gauges, histograms, and summaries—enabling
accurate instrumentation of application behavior and performance across diverse runtime environments such as \ac{JVM} and native platforms.

Central to the architecture is the \texttt{CollectorRegistry}, a thread-safe, platform-agnostic component responsible for managing metric collectors and aggregating their samples.
This registry acts as the single source of truth for all registered metrics and ensures that metric collection can occur concurrently without data races or inconsistencies.

The metric model is structured around an extensible hierarchy of collector abstractions.
The abstract base class \texttt{Collector} defines the core contract for metric collectors, while the \texttt{SimpleCollector} subclass handles metrics with label support by managing child metric instances identified by unique label sets.
Concrete implementations such as \texttt{Counter}, \texttt{Gauge}, \texttt{Histogram}, and \texttt{Summary} extend this hierarchy, encapsulating their specific semantics and data aggregation strategies.

For data exposure, the library includes an exporter module that serializes collected metrics into the Prometheus
text-based exposition format (version 0.0.4)\cite{prometheus_exposition_formats}.
This serialization strictly adheres to the Prometheus scraping protocol, ensuring compatibility with Prometheus servers and related tooling.
To preserve the multiplatform nature of the core library, the HTTP server components responsible for exposing
metrics over the network are isolated into a separate \ac{JVM}-specific module utilizing the Ktor framework.
This modularization prevents platform-specific dependencies from polluting the core library and allows users to selectively include HTTP exposition features as needed.

In summary, the architecture is characterized by:

\begin{itemize}
    \item \textbf{Portability}: Core metric collection logic is platform-independent and compatible with all Kotlin targets.
    \item \textbf{Modularity}: Separation of core logic and platform-specific integrations promotes maintainability and extensibility.
    \item \textbf{Thread Safety}: Use of multiplatform concurrency primitives ensures safe concurrent access to metrics.
    \item \textbf{Standards Compliance}: Exported metrics conform to Prometheus exposition format, guaranteeing interoperability.
\end{itemize}

This architecture provides a scalable and robust foundation for implementing reliable metric instrumentation in Kotlin applications, empowering developers to integrate with Prometheus-based monitoring pipelines effortlessly.

\subsection{Intended Usage and Integration Scenarios}\label{subsec:intended-usage-and-integration-scenarios}

This library is designed to provide a lightweight, idiomatic Prometheus client implementation for Kotlin
applications, with full support for \ac{KMP}.
It exposes a core \ac{API} for defining, updating, and exporting Prometheus metrics without relying on any platform-specific construct.

The architecture is modular, enabling developers to use only what they need.
This allows the library to be integrated into a wide range of
applications, such as:

\begin{itemize}
    \item \textbf{Kotlin Multiplatform background services}, including command-line tools or daemon-like processes, where metrics can be logged or periodically collected by another component.
    \item \textbf{Kotlin/JVM web applications}, which can expose metrics over HTTP by including an additional module that integrates with a web framework (e.g., Ktor).
\end{itemize}

The library aims to support both simple and advanced scenarios.
For example, developers writing a \ac{JVM}-based backend in Ktor can add the HTTP exposition module to expose
metrics at a \texttt{/metrics} endpoint.
Conversely, developers building a multiplatform application that collects metrics locally can use only the core module, avoiding unnecessary dependencies.

This flexibility ensures that the library remains lightweight, portable, and easy to integrate into various environments.


\section{Module Structure}\label{sec:module-structure}

The Prometheus client library is organized into a modular architecture to ensure clean separation of concerns, platform flexibility, and ease of use across different application environments.
Each module encapsulates a specific layer of functionality, ranging from the core metric definitions to optional platform-specific integrations for HTTP exposition.
The current structure consists of the following modules:

\begin{itemize}
    \item \textbf{core} --- A \ac{KMP} module containing the core logic of the Prometheus client library.
    It includes the metric types (e.g., \texttt{Counter}, \texttt{Gauge}), the registry mechanism, the text
    exposition formatter, and the \ac{DSL} for defining metrics.
    This is the only module that is \ac{KMP}-compatible and targets multiple platforms.

    \item \textbf{ktor} --- A \ac{JVM}-only module that integrates the core library with the Ktor web framework.
    It provides a route handler that exposes the registered metrics via an HTTP endpoint (typically at \texttt{/metrics}). Additionally, it automatically registers useful server-side metrics such as request counts, durations, and response statuses, giving developers immediate visibility into their application's performance.

    \item \textbf{http} --- A lightweight \ac{JVM}-only module that provides a minimal HTTP server for metrics
    exposition, independent of any web framework.
    It is specifically designed for users who want to expose metrics in environments that either lack an HTTP server or rely on unsupported frameworks.
    This makes it ideal for background workers, command-line tools, or embedded applications where adopting a full web stack would be excessive.

    \item \textbf{ktor-example} --- A self-contained demonstration module showing how to integrate the library in a real Ktor web application.
    This module is fully Dockerized and includes a complete monitoring stack, featuring Prometheus and Grafana containers for out-of-the-box metrics visualization.
    Additionally, it contains load simulation logic that periodically generates synthetic requests, allowing users to observe dynamic metric changes in real time.

    \item \textbf{ktor-example/docker-compose.yml} --- Provides the configuration to spin up the example service alongside Prometheus and Grafana.
    The Prometheus configuration is preloaded to scrape the Ktor application, and Grafana dashboards are preconfigured to visualize common metrics.
\end{itemize}

\begin{figure}[h]
    \begin{lstlisting}
          +-------+
          | core  |
          +-------+
          /       \
      +-----+    +-----+
      |ktor |    |http |
      +-----+    +-----+
         |
    +-----------+
    |ktor-example|
    +-----------+
    \end{lstlisting}
    \caption{Module dependencies}
\end{figure}

This modular design ensures that users only depend on the components they need.
For example, a \ac{KMP} application targeting native platforms can rely solely on \texttt{core}, while a JVM backend
can optionally include \texttt{ktor} or \texttt{http} depending on its architectural requirements.
The \texttt{ktor-example} module provides a practical blueprint for integration but is not intended to be used as a dependency.


\section{Metric Model}\label{sec:metric-model}

The metric model of the Prometheus Kotlin client library provides abstractions for the four fundamental Prometheus metric types: \texttt{Counter}, \texttt{Gauge}, \texttt{Histogram}, and \texttt{Summary}. Each type has its specific semantics and API designed to facilitate common monitoring tasks in an idiomatic Kotlin manner.

\subsection{Metric Types Overview}

\begin{itemize}
    \item \textbf{Counter:} Represents a cumulative metric that only increases, typically used to count discrete events such as requests, errors, or tasks completed.

    \item \textbf{Gauge:} Represents a metric that can increase and decrease arbitrarily, suitable for measuring instantaneous values such as temperature, memory usage, or active sessions.

    \item \textbf{Histogram:} Records observations within configurable buckets to provide a frequency distribution of values, often used for request durations or response sizes.

    \item \textbf{Summary:} Provides streaming quantile estimations, allowing tracking of percentiles (e.g., median, 95th percentile) over a sliding time window.
\end{itemize}

\subsection{Class Hierarchy and Relationships}

The metric types derive from common abstract base classes to promote code reuse and consistent behavior:

\begin{verbatim}
          +------------------+
          |    Collector     |  (abstract base class)
          +------------------+
                  |
          +------------------+
          | SimpleCollector  |  (supports labeled metrics)
          +------------------+
            /      |        \
    +----------+ +-------+ +----------+
    | Counter  | | Gauge | | Histogram|
    +----------+ +-------+ +----------+
                  |
              +---------+
              | Summary | (direct subclass of Collector)
              +---------+
\end{verbatim}

All core metric types inherit the \texttt{labels(vararg labelValues: String)} method from their base classes, allowing them to create and retrieve labeled child metrics.
This feature enables dimensionality in metrics by associating key-value label pairs with each observation.

\subsection{Detailed Metric Descriptions}\label{subsec:detailed-metric-descriptions}

\paragraph{Counter}

A \texttt{Counter} is a cumulative metric that can only increase or be reset to zero on restart.
It supports labeled child instances to track counts across different dimensions (e.g., HTTP method and status code).

\begin{itemize}
    \item \texttt{inc()} — Increment the counter by 1.
    \item \texttt{inc(amount: Double)} — Increment the counter by a specified positive amount.
    \item \texttt{clear()} — Reset all labeled children to zero.
\end{itemize}

Example usage:

\begin{lstlisting}
val errors = counter("errors_total") {
    help("Total number of errors")
    labelNames("type")
}

errors.labels("database").inc()
errors.labels("network").inc(2.0)
\end{lstlisting}

\paragraph{Gauge}

A \texttt{Gauge} represents a value that can go up and down, allowing tracking of current states or values that fluctuate.

\begin{itemize}
    \item \texttt{inc()} — Increment the gauge by 1.
    \item \texttt{inc(amount: Double)} — Increment the gauge by a specified amount.
    \item \texttt{dec()} — Decrement the gauge by 1.
    \item \texttt{dec(amount: Double)} — Decrement the gauge by a specified amount.
    \item \texttt{set(value: Double)} — Set the gauge to an explicit value.
    \item \texttt{clear()} — Reset all labeled children.
\end{itemize}

Example usage:

\begin{lstlisting}
val memoryUsage = gauge("memory_usage_bytes") {
    help("Current memory usage in bytes")
}

memoryUsage.set(150_000_000)
memoryUsage.inc(1024)
memoryUsage.dec(512)
\end{lstlisting}

\paragraph{Histogram}

A \texttt{Histogram} samples observations into configurable buckets to represent the frequency distribution of observed values.

\begin{itemize}
    \item \texttt{observe(value: Double)} — Record an observation.
    \item \texttt{clear()} — Reset all recorded data.
\end{itemize}

Histograms automatically count the number of observations and maintain a sum of all observed values to calculate averages.

Example usage:

\begin{lstlisting}
val requestLatency = histogram("http_request_duration_seconds") {
    help("HTTP request latency in seconds")
    buckets(0.1, 0.3, 1.5, 10.0)
}

requestLatency.observe(0.42)
requestLatency.observe(1.2)
\end{lstlisting}

\paragraph{Summary}

A \texttt{Summary} provides streaming quantile estimation, which is useful to calculate percentiles over a sliding time window.

\begin{itemize}
    \item \texttt{observe(value: Double)} — Record a sample observation.
    \item \texttt{clear()} — Reset all data.
\end{itemize}

Example usage:

\begin{lstlisting}
val responseSizes = summary("http_response_size_bytes") {
    help("HTTP response sizes in bytes")
}

responseSizes.observe(512)
responseSizes.observe(1024)
\end{lstlisting}

This metric model provides a consistent and extensible foundation for instrumenting Kotlin applications in a way that aligns with Prometheus best practices.
By leveraging labeled child instances via the inherited \texttt{labels()} method and exposing idiomatic APIs for incrementing, setting, or observing values, the library enables rich and precise metric collection.

\section{DSL and Usability}\label{sec:dsl-and-usability}

To provide an idiomatic and fluent \ac{API} for defining metrics, the library leverages Kotlin’s \ac{DSL} capabilities
combined with builder classes.
While the internal implementation uses builder classes such as \texttt{CounterBuilder} and \texttt{GaugeBuilder} to
encapsulate metric configuration, the user-facing \ac{API} exposes concise \ac{DSL}-style functions that hide the
builder pattern behind expressive Kotlin lambda syntax.

For instance, the \texttt{counter} function allows users to define and register a counter metric in a clear and declarative way:

\begin{figure}[h]
    \begin{lstlisting}
val requests = counter("http_requests_total") {
    help("Total HTTP requests")
    labelNames("method", "status")
    includeCreatedSeries(true)
}
    \end{lstlisting}
    \caption{Counter definition - Prometheus Kotlin library}
\end{figure}

This approach provides several benefits:

\begin{itemize}
    \item \textbf{Simplicity}: Users do not interact directly with builder classes, reducing boilerplate and lowering the barrier to entry.
    \item \textbf{Expressiveness}: The Kotlin lambda with receiver enables a readable, structured way to configure metrics.
    \item \textbf{Flexibility through Defaults}: Configuration fields have sensible defaults, allowing users to specify only the parameters they care about.
    \item \textbf{Optionality}: Most configuration steps are optional, and Kotlin’s support for default and named
    arguments makes the \ac{API} easy to use without requiring verbose setup.
\end{itemize}

In contrast, defining metrics in the Prometheus Java library typically requires more verbose, imperative code that
directly manipulates
builders or configuration objects.
For example, a counter metric might be defined as:


\begin{figure}[h]
    \begin{lstlisting}
Counter.Builder builder = Counter.build()
    .name("http_requests_total")
    .help("Total number of HTTP requests")
    .labelNames("method", "status").register();
    \end{lstlisting}
    \caption{Counter definition - Prometheus Java library}
\end{figure}

This comparison highlights how Kotlin’s language features enable a more elegant and developer-friendly \ac{API}
surface, promoting better usability and reducing common errors.

Similarly, other metric types such as gauges are created using analogous \ac{DSL} functions:

\begin{figure}[h]
    \begin{lstlisting}
val memoryUsage = gauge("memory_usage_bytes") {
    help("Current memory usage in bytes")
    unit("bytes")
    clock(customClock)
}
    \end{lstlisting}
    \caption{Gauge definition - Prometheus Kotlin library}
\end{figure}

Overall, this layered design—with internal builder classes and external \ac{DSL} functions—provides a clear separation
of concerns between metric construction and user interaction, enhancing modularity and maintainability while embracing Kotlin idioms for a modern instrumentation library.


\section{Collector Registry}

\lipsum[1]

\section{Exposition}

\lipsum[1]

% Chapter 5
    \chapter{Implementation Details} \label{cap:implementation}


\section{Metric Types and API Design}
\lipsum[1]


\section{Collector Registration and Logic}
\lipsum[1]


\section{Ktor Integration}
\lipsum[1]


% Chapter 6
    \chapter{Evaluation and Limitations} \label{cap:evaluation}

\section{Functional Validation}

The Kotlin Prometheus client was functionally validated against the \textit{OpenMetrics} specification~\cite{openmetrics-spec}, ensuring compliance with required formatting rules, metadata emission, and semantic conventions.

To validate correctness, unit tests were written to verify:
\begin{itemize}
  \item Label handling and metric uniqueness.
  \item Output formatting according to OpenMetrics text exposition format.
  \item Correct behavior of metric operations (\texttt{inc}, \texttt{get}, \texttt{set}) across gauges, counters, histograms, and summaries.
  \item Accurate quantile configuration and observation bucketing for summaries and histograms.
  \item Safe concurrent access using coroutines and \texttt{atomicfu} primitives.
\end{itemize}

Support for histograms includes both linear and exponential bucket configuration. Summary metrics were validated for configurable quantile tracking and tolerances. The client was successfully scraped using a Prometheus server, confirming its compatibility with external collectors and exporters.

\section{Performance Considerations}

The Kotlin client prioritizes coroutine-friendly concurrency and safe metric mutation over raw throughput. As shown in benchmarking (see Chapter~\ref{ch:architecture}), read-heavy operations (\texttt{get}) perform exceptionally well and sometimes surpass the Java client in throughput.

Write-heavy operations such as \texttt{inc()}, \texttt{set()}, and \texttt{observe()} demonstrate slightly lower performance due to coroutine scheduling and atomic synchronization overhead. Histograms and summaries involve more complex logic (e.g., bucketing, rank approximation), which adds some cost to observation recording. However, the performance remains acceptable and predictable under load.

Despite this overhead, the client maintains stable performance under high concurrency thanks to:
\begin{itemize}
  \item Non-blocking metric access patterns.
  \item Scoped coroutine isolation for metric operations.
  \item Use of lock-free atomic primitives.
\end{itemize}

\section{Comparison with Other Clients}

When compared to the official Java Prometheus client:
\begin{itemize}
  \item The Kotlin client has a more idiomatic API tailored to Kotlin, with extension functions, lambdas, and DSL-style builders.
  \item Coroutine support enables suspendable metric operations and better alignment with modern Kotlin concurrency models.
  \item Histograms and summaries follow a simpler configuration model while remaining fully compliant with OpenMetrics.
  \item Mutation operations are slightly slower due to Kotlin's atomic wrappers instead of Java's \texttt{LongAdder}.
\end{itemize}

The inclusion of histogram and summary support brings metric coverage on par with the Java client, while preserving the Kotlin-first design philosophy. The implementation was validated against expected output structures and snapshot comparisons with Java-based benchmarks.

\section{Known Limitations}

While the Kotlin Prometheus client provides robust core features and coroutine support, some limitations remain:
\begin{itemize}
  \item \textbf{Limited metric types:} The client currently supports only core metric types—\texttt{counter}, \texttt{gauge}, \texttt{histogram}, and \texttt{summary}. More specialized metrics such as \texttt{info}, \texttt{state\_set}, or \texttt{enum} are not implemented.
  \item \textbf{No use of \texttt{LongAdder}:} The client avoids using JVM-specific classes like \texttt{LongAdder}, which offer superior write throughput in high-concurrency environments. Instead, it relies on portable atomic primitives to ensure coroutine safety, at the cost of some raw performance.
  \item \textbf{No PushGateway support:} Push-based metric delivery to Prometheus (e.g., via PushGateway) is not currently supported.
\end{itemize}

Future enhancements could address these limitations by:
\begin{itemize}
  \item Adding support for PushGateway integration.
  \item Extending metric coverage to include specialized types such as \texttt{info} and \texttt{state\_set}.
  \item Exploring conditional use of \texttt{LongAdder} via Java interop for high-performance counters.  \item Supporting dynamic histogram configuration.
\end{itemize}



% Chapter 7
    \chapter{Future Work} \label{cap:future}

\section{Expanding Ecosystem Integrations}
While the current Prometheus client supports integration with Ktor, future development should focus on extending support to other popular server frameworks and platforms within the Kotlin ecosystem. This includes native support for:
\begin{itemize}
    \item \textbf{Spring Boot (Kotlin DSL):} Enabling Prometheus metrics registration via auto-configuration and Spring Actuator endpoints.
    \item \textbf{Micronaut and Koin:} Providing lightweight instrumentation tailored for dependency-injection-heavy applications.
    \item \textbf{Android and Kotlin Multiplatform:} Extending observability tooling to mobile or cross-platform environments where Prometheus scraping is less common, potentially using push-based collection or local aggregators.
\end{itemize}

Such integrations would promote broader adoption by reducing the friction for developers building on top of different stacks.

\section{Improve Integration with Well-Known Kotlin Frameworks}
The current Ktor integration demonstrates the feasibility of lightweight, coroutine-friendly metric collection. However, more ergonomic integrations can be achieved by:
\begin{itemize}
    \item Introducing \textbf{Kotlin-style DSLs} for declaratively registering metrics at application startup.
    \item Offering built-in support for Ktor’s typed routing, allowing route-specific metrics without requiring manual normalization or labeling.
    \item Enabling pluggable interceptors to monitor application features like database access, authentication, or external API calls.
\end{itemize}

Providing idiomatic APIs that reflect Kotlin’s language features (extension functions, delegation, lambdas) would help developers embed metrics more naturally into their applications.

\section{Automatic JVM Metrics Collection}
A key enhancement for production-readiness is enabling automatic collection of runtime-level metrics from the JVM environment. This includes:
\begin{itemize}
    \item \textbf{Garbage collection and memory usage:} Exposing heap/non-heap statistics and GC pause durations.
    \item \textbf{Thread metrics:} Monitoring live threads, daemon threads, and blocked thread counts.
    \item \textbf{Class loading and compilation metrics:} Tracking class definitions, unloads, and JIT compilation time.
\end{itemize}

These metrics are critical for performance tuning and operational visibility, especially in cloud-native deployments. Leveraging the standard \texttt{java.lang.management} API, these metrics can be collected automatically on JVM startup and registered in the Prometheus client’s registry.

Combining application-level and system-level metrics would provide a complete observability picture, aligning the client with the capabilities offered by the official Prometheus Java client.


% Chapter 8
    \chapter{Conclusion} \label{cap:conclusion}


\section{Summary of Contributions}
\lipsum[1]


\section{Lessons Learned}
\lipsum[1]


\section{Final Remarks}
\lipsum[1]

% References
    \bibliographystyle{plain}
    \bibliography{refs}
    \addcontentsline{toc}{chapter}{References}

    \appendix
    %
% Appendix 1
%
\chapter{Appendix} \label{ap:exemplo}
\lipsum[14-16]

\end{document} 