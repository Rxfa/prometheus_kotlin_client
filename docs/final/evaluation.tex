\chapter{Evaluation and Limitations} \label{cap:evaluation}

\section{Functional Validation}

The Kotlin Prometheus client was functionally validated against the \textit{OpenMetrics} specification~\cite{openmetrics-spec}, ensuring compliance with required formatting rules, metadata emission, and semantic conventions.

To validate correctness, unit tests were written to verify:
\begin{itemize}
  \item Label handling and metric uniqueness.
  \item Output formatting according to OpenMetrics text exposition format.
  \item Correct behavior of metric operations (\texttt{inc}, \texttt{get}, \texttt{set}) across gauges and counters.
  \item Safe concurrent access using coroutines and \texttt{atomicfu} primitives.
\end{itemize}

The client also successfully scraped and exposed metrics using a basic Prometheus setup, confirming compatibility with Prometheus servers and exporters.

\section{Performance Considerations}

The Kotlin client prioritizes coroutine-friendly concurrency and safe metric mutation over raw throughput. As shown in benchmarking (see Chapter~\ref{ch:architecture}), read-heavy operations (\texttt{get}) perform exceptionally well and sometimes surpass the Java client in throughput.

However, write-heavy operations such as \texttt{inc()} or \texttt{set()} demonstrate slightly lower performance compared to the Java implementation. This is primarily due to the use of coroutine dispatchers and atomic operations, which add safety and idiomatic Kotlin integration at the cost of some speed.

Despite the overhead, the Kotlin client maintains stable and predictable performance under concurrent load, benefiting from structured concurrency, non-blocking updates, and coroutine scope isolation.

\section{Comparison with Other Clients}

When compared to the official Java Prometheus client:
\begin{itemize}
  \item The Kotlin client has a **simpler and more idiomatic API** tailored to Kotlin, reducing boilerplate and making use of extension functions, lambdas, and builders.
  \item It integrates **native coroutine support**, allowing suspendable metric operations, which is not supported in Java clients.
  \item **Performance for metric reads** is comparable or better in some cases. However, **metric mutation operations** are slightly slower, as described in the previous section.
  \item The Kotlin implementation uses **atomic primitives** instead of Java’s \texttt{LongAdder}, which performs better in high-throughput, multi-threaded scenarios.
\end{itemize}

Additionally, the Kotlin client’s adherence to the OpenMetrics spec ensures output compatibility and Prometheus observability tooling interoperability.

\section{Known Limitations}

While the Kotlin Prometheus client provides robust features and coroutine support, some limitations are acknowledged:
\begin{itemize}
  \item \textbf{Lack of LongAdder-based counters:} Java's high-performance atomic counters are not used due to Kotlin/JVM abstraction, impacting mutation throughput.
  \item \textbf{Limited metric types:} Only core types like counters and gauges are currently implemented. Summaries and histograms are not yet supported.
  \item \textbf{Lack of pull-based HTTP exposition:} The client focuses only on in-memory collection and formatting; an HTTP layer must be integrated externally.
  \item \textbf{No static configuration DSL:} Unlike some Kotlin libraries, the client does not yet provide a declarative DSL to register and expose metrics.
\end{itemize}

Future enhancements could address these limitations by:
\begin{itemize}
  \item Using Java interop to leverage \texttt{LongAdder} when appropriate.
  \item Implementing more metric types (e.g., summary, histogram).
  \item Providing an HTTP exposition server component.
  \item Adding a structured DSL for metric configuration and registration.
\end{itemize}
