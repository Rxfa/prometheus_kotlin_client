\chapter{Implementation Details} \label{cap:implementation}


%---------------------------
% Counter
%---------------------------

\section{Counter Metric Implementation}

The \texttt{Counter} metric implementation adheres to the Prometheus convention of representing monotonically increasing values, typically used to count events such as HTTP requests, errors, or completed tasks.

\subsection{Construction and Registration}
The \texttt{counter} DSL function allows users to build and register counters in a declarative manner:
\begin{verbatim}
val requests = counter("http_requests_total") {
    help("Total HTTP requests")
    labelNames("method", "status")
}
\end{verbatim}
Internally, this utilizes a \texttt{CounterBuilder} which constructs and registers a \texttt{Counter} instance.




\subsection{Metric Semantics and Structure}
The \texttt{Counter} class extends a generic \texttt{SimpleCollector} and supports optional labels, units, and the inclusion of a \texttt{\_created} time series. The final metric name is normalized to include a \texttt{\_total} suffix, and optionally a unit suffix:
\begin{verbatim}
if (unit.isNotBlank()) {
    metricName = "${metricName}_${unit}"
}
return "${metricName}_total"
\end{verbatim}



%Atomic Value Handling

\subsection{Atomic Value Handling}

Due to the lack of native atomic floating-point primitives in Kotlin/Native, counter values in this implementation are managed using the \texttt{kotlinx.atomicfu} library. Each labeled instance of the counter is encapsulated within a \texttt{Child} class, where the internal value is stored as a raw bit representation of a \texttt{Double} using an atomic \texttt{Long}.

\paragraph{Why not use \texttt{AtomicRef<Double>}?}  
At first glance, it may seem sufficient to store a \texttt{Double} value using an \texttt{AtomicRef}. However, this approach introduces inefficiencies in a concurrent environment. Since Kotlin/Native lacks atomic increment operations for \texttt{Double}, using an atomic reference would require manual compare-and-set (CAS) loops to ensure thread safety, as shown below:

\begin{verbatim}
while (true) {
    val current = value.value
    val updated = current + amount
    if (value.compareAndSet(current, updated)) {
        return
    }
}
\end{verbatim}

This approach may lead to excessive CPU usage, especially under high contention, due to repeated CAS failures and retries.

\paragraph{Raw Bits Workaround}

Instead, the implementation stores the \texttt{Double} value as its raw IEEE 754 bit representation in a \texttt{Long}. This allows atomic updates using the efficient \texttt{updateAndGet} operation provided by \texttt{atomicfu}:

\begin{verbatim}
value.updateAndGet { currentBits ->
    val current = Double.fromBits(currentBits)
    val updated = current + amount
    updated.toBits()
}
\end{verbatim}

This approach ensures atomicity in a single operation, avoiding busy-wait loops and improving performance in concurrent scenarios. It also maintains correct floating-point semantics without introducing data races or synchronization issues.


\begin{verbatim}
private var value = atomic(0.0.toRawBits())
\end{verbatim}



%Other Viable Option

\paragraph{Other viable option}
Since there are multiple different uses for counters, we could optimise it for other uses.

For example:
In cases where most of the increments would be only 1
We could use two instances, one for \texttt{AtomicLong} and another one for the Double representation in rawbits \texttt{AtomicLong('Double in RawBits')} for the increments.


\begin{verbatim}
private var valueLong = atomic(0L)
private var valueDouble = atomic(0.0.toRawBits())
\end{verbatim}

And when we used inc() it would only increment the valueLong like this:
\begin{verbatim}
valueLong.incrementAndGet()
\end{verbatim}


But when used get() it would return the sum of both values:
\begin{verbatim}
return Double.fromBits(valueDouble.value) + valueLong.value
\end{verbatim}

A future enhancement could explore this dual-storage approach to a another type of Counter inside of our library.


%Concurrency Model

\subsection{Concurrency Model}
All increment operations are performed within the \texttt{Dispatchers.Default} coroutine context to ensure scalability across multiple threads. This design leverages Kotlin coroutines to provide structured concurrency and minimize blocking.

We opted out of using \texttt{Dispatchers.IO} or \texttt{Dispatchers.Unconfined} to avoid potential performance bottlenecks and ensure that counter updates are handled efficiently in a thread-safe manner.

Based on the nature of the counter it only made sense to make it available in the \texttt{Default} dispatcher, which is optimized for CPU-bound tasks. This choice aligns with the typical use cases of counters, such as counting requests or events, which are often CPU-intensive operations.

%Label Handling and Sample Collection

\subsection{Label Handling and Sample Collection}
Counters may be labeled using the \texttt{labels(...)} method. The collector stores a map from label sets to their respective \texttt{Child} instances. During metric scraping, the \texttt{collect()} method emits one sample per label set. If the \texttt{includeCreatedSeries} flag is enabled, an additional sample with the \texttt{\_created} suffix is also emitted:
\begin{verbatim}
val createdSeriesName = name.removeSuffix("_total") + "_created"
\end{verbatim}


In this implementation, the \texttt{collect()} function on the \texttt{Counter} class gathers all current values of each labeled or unlabeled counter instance and packages them as \texttt{Sample} objects. These samples are then grouped into a \texttt{MetricFamilySamples} data structure, which is Prometheus's expected format for exporting metrics.

Each call to \texttt{collect()} produces one or more samples depending on how many label sets (i.e., unique combinations of labels) have been used to increment the counter. If \texttt{includeCreatedSeries} is enabled, an additional \texttt{\_created} time series is emitted for each label set, indicating when the metric series was initialized.

\vspace{0.5em}
\noindent
\textbf{Example Sample Output}:
\begin{verbatim}
http_requests_total{method="GET", status="200"}  123.0
http_requests_created{method="GET", status="200"}  1.650e+09
\end{verbatim}

\subsection{Labeled vs Unlabeled Counter Incrementation}

Prometheus counters in this library support both labeled and unlabeled usage. The difference between these modes lies in the ability to partition metrics by key-value identifiers (labels).

\begin{itemize}
  \item \textbf{Unlabeled Counter:}
    \begin{verbatim}
    val errors = counter("error_total") { help("Total errors") }
    errors.inc()
    \end{verbatim}
    This counter tracks a single, global value without distinguishing between contexts.

  \item \textbf{Labeled Counter:}
    \begin{verbatim}
    val requests = counter("http_requests_total") {
        help("Total HTTP requests")
        labelNames("method", "status")
    }
    requests.labels("GET", "200").inc()
    \end{verbatim}
    This counter tracks separate values for each unique combination of label values.
\end{itemize}

Using labels is powerful because it enables fine-grained observability. For example, instead of one total count of HTTP requests, you can track how many were successful GETs versus failed POSTs. However, each unique label set creates a new time series, which can increase memory usage and complexity. When fine granularity is not needed, unlabeled counters offer a simpler and more efficient alternative.

%Exception Counting Utility

\subsection{Exception Counting Utility}
To facilitate observability in failure-prone operations, the implementation provides utility extensions to count exceptions:
\begin{verbatim}
counter.countExceptions(IOException::class) {
    // risky operation
}
\end{verbatim}
These suspend functions catch specified exception types and increment the counter accordingly. Both the \texttt{Counter} and its \texttt{Child} instances support this pattern.



%---------------------------
% GAUGE
%---------------------------


\section{Gauge Metric Implementation}

The \texttt{Gauge} metric implementation aligns with Prometheus semantics for representing instantaneous values that can arbitrarily go up and down. It is suitable for metrics such as current memory usage, number of active connections, or custom user-defined values.

\subsection{Construction and Registration}
The \texttt{gauge} DSL function enables declarative creation and registration of gauge metrics:
\begin{verbatim}
val temperature = gauge("room_temperature_celsius") {
    help("Current room temperature")
    labelNames("room")
}
\end{verbatim}
Internally, this uses a \texttt{GaugeBuilder} to configure and instantiate the \texttt{Gauge} metric. Label names and help descriptions are passed through the builder for validation and registration.





\subsection{Metric Semantics and Structure}
The \texttt{Gauge} class extends the generic 
\texttt{SimpleCollector} and supports optional labels, 
an optional unit, and the inclusion of a \texttt{\_created} series if desired. 
The metric name is preserved exactly as provided, with no suffix normalization (unlike counters that enforce a \texttt{\_total} suffix).

Each \texttt{Gauge} tracks a single floating-point value per label set, which may be incremented, decremented, or explicitly set. A convenience child without labels is always created if no label names are provided.


%Atomic Value Handling

\subsection{Atomic Value Handling}
Similar to the counter, the gauge uses the \texttt{kotlinx.atomicfu} library for thread-safe atomic updates of floating-point values. Each labeled instance is encapsulated in a \texttt{Child} class, which stores the internal value as a raw bitwise representation of a \texttt{Double} using an \texttt{AtomicLong}:

\begin{verbatim}
private var value = atomic(0.0.toRawBits())
\end{verbatim}

All value updates (e.g., \texttt{inc()}, \texttt{dec()}, \texttt{set()}) operate via atomic \texttt{updateAndGet()} transformations to ensure correctness and eliminate the need for locks:

\begin{verbatim}
value.updateAndGet { currentBits ->
    val current = Double.fromBits(currentBits)
    val updated = current + amount
    updated.toBits()
}
\end{verbatim}

This approach allows safe floating-point arithmetic in concurrent environments, while maintaining high performance and avoiding CAS loop pitfalls.

%Other Viable Option

\paragraph{Other viable option}
Gauges differ from counters in that they support both incrementing and decrementing, as well as setting values directly. This broader range of use cases opens the door to additional optimization strategies.

For instance, in scenarios where gauges are used primarily to reflect binary, or to increment/decrement by whole numbers in tight loops (e.g., tracking open connections), we could explore separating the fast-path operations from general-purpose floating-point updates.

A potential optimization would involve using two atomic values: one \texttt{AtomicLong} for whole number delta-based changes, and one \texttt{AtomicLong} for the base \texttt{Double} value stored in raw bits:

\begin{verbatim}
private var valueLong = atomic(0L)
private var valueDouble = atomic(0.0.toRawBits())
\end{verbatim}

In this model:

    \texttt{inc()} and \texttt{dec()} by \texttt{1.0} would operate exclusively on \texttt{valueLong}:
    \begin{verbatim}
    valueLong.incrementAndGet()
    \end{verbatim}

    \texttt{set()} would overwrite \texttt{valueDouble}, resetting the gauge’s baseline:
    \begin{verbatim}
    valueDouble.value = newValue.toRawBits()
    valueLong.value = 0L
    \end{verbatim}

    \texttt{get()} would compute the current gauge value as:
    \begin{verbatim}
    return Double.fromBits(valueDouble.value) + valueLong.value
    \end{verbatim}

This dual-storage approach would preserve general-purpose floating-point precision for arbitrary updates while optimizing for frequent whole-number changes, improving performance and reducing contention in certain workloads.

While not currently implemented, this model could serve as the basis for a specialized \texttt{IntOptimizedGauge} or \texttt{FastGauge} variant in future versions of the library, tailored to specific use cases with tighter numerical requirements.


%Concurrency Model


\subsection{Concurrency Model}
Gauge operations are suspend functions and are always executed in the \texttt{Dispatchers.Default} coroutine context. This allows for structured concurrency while ensuring that value modifications do not block the main thread:

\begin{verbatim}
withContext(Dispatchers.Default) {
    value.updateAndGet { ... }
}
\end{verbatim}

This model makes gauges suitable for high-concurrency applications such as web servers or background task schedulers.

\subsection{Tracking and Duration Utilities}
Gauges include utility functions for tracking the number of concurrent operations and measuring execution time:

\paragraph{In-Flight Task Tracking}

\begin{verbatim}
gauge.track {
    performWork()
}
\end{verbatim}

The \texttt{track()} function increments the gauge when the block begins and decrements it after the block completes, making it ideal for concurrency observability.

\paragraph{Duration Measurement}

\begin{verbatim}
gauge.setDuration {
    expensiveCall()
}
\end{verbatim}

This utility measures how long a code block takes and sets the gauge to the elapsed time in seconds. This can be used with or without labels, and supports both \texttt{Gauge} and \texttt{Gauge.Child} instances.

\subsection{Label Handling and Sample Collection}
Gauges support labels via the \texttt{labels(...)} method. A \texttt{Child} instance is maintained for each unique combination of label values. During metric collection, one sample is emitted per label set:

\begin{verbatim}
val samples = mutableListOf<Sample>()
for ((labels, child) in childMetrics) {
    samples += Sample(name, labelNames, labels, child.get())
}
\end{verbatim}

Each sample includes its respective label values and the current gauge reading. Label handling logic is inherited from \texttt{SimpleCollector}, ensuring consistency across all metric types in the library.

\subsection{What Are Samples in Gauge Metrics?}

In Prometheus, a \textit{sample} is a single point of measurement containing a metric name, optional labels, and a value. For a \texttt{Gauge}, which represents a numerical value that can increase or decrease, samples are used to report the current value at collection time.

This implementation stores one or more \texttt{Sample} instances per gauge, depending on the number of label combinations used. The \texttt{collect()} method gathers these samples by iterating through all registered \texttt{Child} instances, each tied to a distinct label set.

Unlike counters, gauges do not enforce monotonicity and can go up or down depending on the application's state (e.g., number of in-flight requests, memory usage, or temperature). Each sample accurately reflects the current value for its label set at the moment of collection.

\vspace{0.5em}
\noindent
\textbf{Example Sample Output}:
\begin{verbatim}
room_temperature_celsius{room="kitchen"} 22.5
room_temperature_celsius{room="bedroom"} 19.0
\end{verbatim}

\subsection{Labeled vs Unlabeled Gauge Incrementation}

Like counters, gauges can be used in labeled and unlabeled modes. The primary distinction lies in how they track different instances of the same metric.

\begin{itemize}
  \item \textbf{Unlabeled Gauge:}
    \begin{verbatim}
    val temperature = gauge("room_temperature_celsius") {
        help("Room temperature")
    }
    temperature.set(22.5)
    \end{verbatim}
    This gauge tracks a single, global value, without any differentiation between contexts.

  \item \textbf{Labeled Gauge:}
    \begin{verbatim}
    val temperature = gauge("room_temperature_celsius") {
        help("Room temperature")
        labelNames("room")
    }
    temperature.labels("kitchen").set(22.5)
    temperature.labels("bedroom").set(19.0)
    \end{verbatim}
    Labeled gauges maintain separate values for each unique combination of label values. This is useful when the same type of measurement must be tracked in different categories, such as rooms, users, or endpoints.
\end{itemize}

Labeled gauges provide granular observability but require careful management of label cardinality to avoid performance issues. Unlabeled gauges are simpler and more efficient when only one measurement context is needed.



\subsection{Time-Based Gauge Values}
A gauge can also be set to the current system time in Unix seconds using the \texttt{setToCurrentTime()} method. This is useful for tracking freshness or heartbeat-style metrics:

\begin{verbatim}
gauge.setToCurrentTime()
\end{verbatim}

Internally, this uses the injected \texttt{Clock} instance (defaulting to \texttt{Clock.System}) to fetch the current time and update the gauge accordingly.


%---------------------------
% HISTOGRAM
%---------------------------


\section{Histogram Metric Implementation}

The \texttt{Histogram} metric implementation models the distribution of observations by tracking both their count and the sum of all observed values. It uses a fixed set of configurable buckets to group values and is well-suited for timing operations, request latencies, payload sizes, etc.

\subsection{Construction and Registration}

Users can construct histograms declaratively using the provided DSL functions:

\begin{verbatim}
val latency = histogram("http_request_duration_seconds") {
    help("Request latency in seconds")
    labelNames("method", "endpoint")
}
\end{verbatim}

Multiple bucket strategies are available:
\begin{itemize}
    \item \texttt{histogramBuckets(...)} – custom bucket boundaries
    \item \texttt{linearHistogramBuckets(...)} – evenly spaced buckets
    \item \texttt{exponentialHistogramBuckets(...)} – exponentially increasing bucket widths
\end{itemize}

Each function delegates to \texttt{HistogramBuilder}, which validates and registers the final \texttt{Histogram} instance.

\subsection{Metric Semantics and Structure}

The \texttt{Histogram} class extends \texttt{SimpleCollector} and emits three series per label set:
\begin{itemize}
    \item \texttt{\_bucket} – cumulative count of observations per bucket
    \item \texttt{\_sum} – sum of all observed values
    \item \texttt{\_count} – total number of observations
\end{itemize}

The final bucket is automatically extended to include \texttt{+\(\infty\)} if not explicitly present. The label \texttt{le} (less-than-or-equal) is reserved and disallowed as a user-defined label.

\subsection{Atomic Value Handling}

Each histogram \texttt{Child} maintains:
\begin{itemize}
    \item an atomic sum of all observed values (using \texttt{Double.toRawBits()} stored in an \texttt{AtomicLong}),
    \item a per-bucket array of atomic counters (\texttt{AtomicLong}) tracking the number of observations.
\end{itemize}

Example for sum tracking:
\begin{verbatim}
sum.updateAndGet { bits ->
    val current = Double.fromBits(bits)
    val updated = current + value
    updated.toRawBits()
}
\end{verbatim}

This design ensures atomic updates in concurrent environments without using CAS loops or locks.

\subsection{Bucket Recording Logic}

When a value is observed, it is recorded in the first bucket whose upper bound is greater than or equal to the value. This is done atomically within the \texttt{Dispatchers.Default} context:

\begin{verbatim}
for (i in upperBounds.indices) {
    if (value <= upperBounds[i]) {
        cumulativeCounts[i].getAndIncrement()
        break
    }
}
\end{verbatim}

Each observation also updates the running sum of all values.

\subsection{Timing Utility}

The histogram provides utilities for measuring and recording durations:

\begin{itemize}
    \item \texttt{startTimer()} – returns a \texttt{Timer} that captures duration from start
    \item \texttt{observeDuration()} – records elapsed time in seconds
    \item \texttt{time\{...\}} – runs a block and records execution time
\end{itemize}

\begin{verbatim}
val timer = histogram.labels("GET", "/api").startTimer()
// ... do something ...
timer.observeDuration()
\end{verbatim}

This makes histograms especially useful for measuring latencies.

\subsection{Concurrency Model}

All observation operations are dispatched on \texttt{Dispatchers.Default} to ensure concurrency safety and performance in multi-threaded environments. No synchronization primitives like mutexes are used; instead, atomic operations provide safe concurrent access.

\subsection{Label Handling and Sample Collection}

Like other collectors, histograms use a map of label sets to \texttt{Child} instances. The \texttt{collect()} function emits:

\begin{itemize}
    \item a \texttt{\_bucket} sample for each bucket and label set,
    \item a \texttt{\_sum} sample with the total observed value,
    \item a \texttt{\_count} sample for the total observation count.
\end{itemize}

\begin{verbatim}
http_request_duration_seconds_bucket{method="GET", endpoint="/"} 4
http_request_duration_seconds_bucket{method="GET", endpoint="/", le="0.5"} 7
http_request_duration_seconds_count{method="GET", endpoint="/"} 11
http_request_duration_seconds_sum{method="GET", endpoint="/"} 3.42
\end{verbatim}

\subsection{Labeled vs Unlabeled Histogram Observation}

Histograms support both labeled and unlabeled usage:

\paragraph{Unlabeled Histogram:}
\begin{verbatim}
val durations = histogram("job_duration_seconds") { help("Job durations") }
durations.observe(1.23)
\end{verbatim}

\paragraph{Labeled Histogram:}
\begin{verbatim}
val durations = histogram("job_duration_seconds") {
    help("Job durations")
    labelNames("job", "status")
}
durations.labels("backup", "success").observe(0.75)
\end{verbatim}

Each label set maintains its own independent observation state. This allows for detailed breakdowns, such as per-endpoint or per-method latency.

\subsection{Histogram State Retrieval}

Each \texttt{Child} exposes its current state through \texttt{get()}, returning a \texttt{ValueHistogram}:

\begin{verbatim}
val state = histogram.labels("GET", "/").get()
println(state.sum)
println(state.buckets)
\end{verbatim}

This is useful for introspection or exporting metrics to other formats.

\subsection{Bucket Strategies and Utilities}

The histogram DSL provides several utility functions to define bucket boundaries that determine how values are grouped into time series. These functions help users choose appropriate bucket layouts for their data, such as fixed-width ranges or logarithmic distributions.

\paragraph{1. \texttt{histogramBuckets(...)} – Custom Buckets}

This method allows users to explicitly define the upper bounds of each bucket as a list of \texttt{Double} values.

\begin{verbatim}
histogramBuckets(listOf(0.1, 0.5, 1.0, 2.5, 5.0, 10.0))
\end{verbatim}

\textbf{Validation Rules}:
\begin{itemize}
    \item The list must be non-empty.
    \item Values must be strictly increasing (sorted and unique).
    \item The final bucket (i.e., +\(\infty\)) is automatically appended and does not need to be included.
\end{itemize}

This gives users full control over the distribution granularity.

\paragraph{2. \texttt{linearHistogramBuckets(...)} – Linearly Spaced Buckets}

This method generates buckets that are equally spaced. It requires:

\begin{itemize}
    \item \texttt{start} – the lower bound of the first bucket.
    \item \texttt{width} – the fixed interval between bucket boundaries.
    \item \texttt{count} – the number of buckets to generate.
\end{itemize}

\begin{verbatim}
linearHistogramBuckets(start = 0.0, width = 1.0, count = 5)
\end{verbatim}

This will generate buckets with upper bounds: \texttt{[1.0, 2.0, 3.0, 4.0, 5.0]}, plus the final \texttt{+\(\infty\)} bucket.

\textbf{Validation Rules}:
\begin{itemize}
    \item \texttt{count} must be positive.
    \item \texttt{width} must be strictly greater than zero.
    \item \texttt{start} may be any finite \texttt{Double} value.
\end{itemize}

This layout is ideal when your values fall within a bounded linear range, such as number of retries or file sizes in MB.

\paragraph{3. \texttt{exponentialHistogramBuckets(...)} – Exponentially Spaced Buckets}

This method produces buckets whose boundaries grow exponentially. It is defined by:

\begin{itemize}
    \item \texttt{start} – the lower bound of the first bucket (must be > 0).
    \item \texttt{factor} – the multiplication factor between successive bucket boundaries (must be > 1.0).
    \item \texttt{count} – the number of buckets to generate.
\end{itemize}

\begin{verbatim}
exponentialHistogramBuckets(start = 0.001, factor = 2.0, count = 5)
\end{verbatim}

This will produce boundaries:
\texttt{[0.001, 0.002, 0.004, 0.008, 0.016]}, plus \texttt{+\(\infty\)}.

\textbf{Validation Rules}:
\begin{itemize}
    \item \texttt{start} must be strictly greater than zero.
    \item \texttt{factor} must be greater than 1.0.
    \item \texttt{count} must be a positive integer.
\end{itemize}

This layout is excellent for timing or latency distributions where small values are common, but outliers are orders of magnitude larger.

\vspace{0.5em}
\noindent
\textbf{Example Use Case:}
\begin{verbatim}
val latency = histogram("request_latency_seconds") {
    help("Request duration in seconds")
    labelNames("path")
    exponentialHistogramBuckets(start = 0.001, factor = 2.0, count = 10)
}
\end{verbatim}

\paragraph{Final Bucket Inclusion}

Regardless of which strategy is used, the implementation always appends a final implicit bucket with upper bound \texttt{+\(\infty\)} to ensure that every possible observation is recorded.



%---------------------------
% SUMMARY
%---------------------------



\section{Summary Metric Implementation}

The \texttt{Summary} metric implementation provides statistical insights into a stream of observations, particularly useful for tracking latencies, response times, or other continuous measurements. It captures the total count, sum of values, and optionally, configured quantiles over a rolling time window.

\subsection{Construction and Registration}

Users can construct summaries either with or without quantiles using the DSL:

\paragraph{Without Quantiles:}
\begin{verbatim}
val latency = summary("http_request_duration_seconds") {
help("Request duration in seconds")
labelNames("method", "endpoint")
}
\end{verbatim}

\paragraph{With Quantiles:}
\begin{verbatim}
val p95 = quantile(0.95, error = 0.01)
val p99 = quantile(0.99, error = 0.005)

val latency = summaryQuantiles("http_request_duration_seconds",
quantiles = listOf(p95, p99)
) {
help("Request duration in seconds")
labelNames("method", "endpoint")
}
\end{verbatim}

Quantiles are configured using the \texttt{quantile()} function, which validates the quantile value and acceptable error margin. The summary builder validates and registers the final \texttt{Summary} instance.

\subsection{Metric Semantics and Structure}

The \texttt{Summary} class extends \texttt{SimpleCollector} and emits the following time series per label set:
\begin{itemize}
\item \texttt{\_quantile} – value at configured quantiles (if any),
\item \texttt{\_sum} – sum of all observed values,
\item \texttt{\_count} – total number of observations.
\end{itemize}

The \texttt{quantile} la bel is automatically appended when quantile values are reported. User-defined labels must not include \texttt{quantile}.

\subsection{Quantile Computation Model}

Quantiles are calculated using a time-decaying window, partitioned into rotating age buckets. Configuration options include:
\begin{itemize}
\item \texttt{maxAgeSeconds} – length of the observation window (default: 60s),
\item \texttt{ageBuckets} – number of time slices within the window (default: 5).
\end{itemize}

The quantile estimation uses a sliding window algorithm that offers trade-offs between accuracy and memory/CPU usage.

\subsection{Atomic Value Handling}

Each \texttt{Child} tracks state independently and atomically:
\begin{itemize}
\item \texttt{count} – atomic counter of observations,
\item \texttt{sum} – atomic floating-point sum using \texttt{Double.toRawBits()} stored in an \texttt{AtomicLong},
\item \texttt{quantilesValues} – optional time-windowed quantile estimator.
\end{itemize}

Example sum update:
\begin{verbatim}
sum.updateAndGet { bits ->
val current = Double.fromBits(bits)
val updated = current + value
updated.toRawBits()
}
\end{verbatim}

\subsection{Observation Logic}

Observations are recorded using the \texttt{observe()} method:

\begin{verbatim}
summary.labels("POST", "/api").observe(1.23)
\end{verbatim}

The update is dispatched on \texttt{Dispatchers.Default} for safe concurrent usage:

\begin{verbatim}
withContext(Dispatchers.Default) {
count.getAndIncrement()
sum.updateAndGet { ... }
quantilesValues?.insert(value)
}
\end{verbatim}

This ensures thread safety without explicit locking.

\subsection{Timing Utility}

Summaries support timing operations via:
\begin{itemize}
\item \texttt{startTimer()} – creates a timer capturing the start timestamp,
\item \texttt{observeDuration()} – measures elapsed time and records it,
\item \texttt{time{...}} – executes a block and records execution duration.
\end{itemize}

\begin{verbatim}
val timer = summary.labels("GET", "/").startTimer()
// ... do work ...
timer.observeDuration()
\end{verbatim}

This is useful for monitoring latencies of code blocks.

\subsection{Concurrency Model}

All metric updates occur inside the \texttt{Dispatchers.Default} coroutine context. Atomic references (\texttt{AtomicLong}) and lock-free algorithms ensure high concurrency without contention.

\subsection{Label Handling and Sample Collection}

Each unique set of labels corresponds to a \texttt{Child} instance. The \texttt{collect()} function emits:

\begin{itemize}
    \item \texttt{\_quantile} series for each defined quantile, 
    \item \texttt{\_sum} series for the total of all observed values, 
    \item \texttt{\_count} series for the total number of observations. 
\end{itemize}


\begin{verbatim}
http_request_duration_seconds_quantile{method="GET", endpoint="/", quantile="0.95"} 0.63
http_request_duration_seconds_count{method="GET", endpoint="/"} 128
http_request_duration_seconds_sum{method="GET", endpoint="/"} 89.5
\end{verbatim}

\subsection{Labeled vs Unlabeled Summary Observation}

Summaries support both labeled and unlabeled usage:

\paragraph{Unlabeled Summary:}
\begin{verbatim}
val summary = summary("job_duration_seconds") { help("Duration of jobs") }
summary.observe(1.2)
\end{verbatim}

\paragraph{Labeled Summary:}
\begin{verbatim}
val summary = summary("job_duration_seconds") {
help("Duration of jobs")
labelNames("job", "status")
}
summary.labels("backup", "success").observe(0.9)
\end{verbatim}

Each label set maintains its own independent rolling state and quantile estimators.

\subsection{Summary State Retrieval}

Each \texttt{Child} exposes its current state via the \texttt{get()} method, returning a \texttt{ValueSummary}:

\begin{verbatim}
val value = summary.labels("GET", "/").get()
println("count: ${value.count}, sum: ${value.sum}, p95: ${value.quantiles?.get(0.95)}")
\end{verbatim}

This is useful for diagnostics or exposing metrics to other sinks.

\subsection{Quantile Validation and Utilities}

Quantiles must be:
\begin{itemize}
\item Between 0.0 and 1.0 (inclusive),
\item Paired with an error margin between 0.0 and 1.0.
\end{itemize}

\begin{verbatim}
val q = quantile(0.99, 0.01)
\end{verbatim}

Attempting to use \texttt{quantile} as a label will result in an \texttt{IllegalStateException}.





\section{Metric Types and API Design}
\lipsum[1]


\section{Collector Registration and Logic}
\lipsum[1]


\section{Ktor Integration}
\lipsum[1]
