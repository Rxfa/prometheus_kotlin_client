\chapter{Problem Statement and Goals} \label{ch:problemdescription}

\section{Identified Limitations}

While Prometheus has become the de facto standard for metrics-based observability in cloud-native systems, its official client libraries are largely written in imperative languages such as Go, Java, and Python. These implementations do not align well with Kotlin’s idiomatic programming model, particularly regarding coroutine support, functional DSLs, and thread-safety via structured concurrency.

The official Java Prometheus client, while usable in Kotlin projects, exposes a verbose and boilerplate-heavy API. Furthermore, it does not integrate naturally with Kotlin coroutines, leading to potential blocking behavior or unsafe metric updates when used improperly in suspendable contexts.

Additionally, Kotlin developers lack a native library that fully embraces Kotlin-specific features like extension functions, scope builders, and coroutine-friendly primitives for metrics definition and recording. This mismatch creates friction when building observability into modern reactive applications using frameworks such as Ktor or Spring WebFlux.

\section{Project Goals}

This project aims to develop a lightweight, idiomatic Prometheus client library written in Kotlin. The primary goals are:

\begin{itemize}
  \item \textbf{Kotlin-first design:} Create a metrics API that aligns with Kotlin’s language features, using builder functions, lambdas, and extension patterns to reduce boilerplate and improve readability.
  \item \textbf{Coroutine safety:} Ensure that metric operations can be safely invoked from within suspending functions and concurrent coroutines, without introducing race conditions or blocking behavior.
  \item \textbf{OpenMetrics compatibility:} Conform to the OpenMetrics exposition format, ensuring compatibility with Prometheus scraping, tooling, and external monitoring pipelines.
  \item \textbf{Core metric support:} Implement the most commonly used metric types—\texttt{counter}, \texttt{gauge}, \texttt{histogram}, and \texttt{summary}—with support for labeling, concurrency, and serialization.
  \item \textbf{Extensibility:} Provide a modular and extensible architecture, allowing future features such as HTTP exposition, JVM metrics integration, or PushGateway support to be added incrementally.
\end{itemize}

\section{Out of Scope}

To maintain focus and deliver a stable MVP, certain features and use cases are intentionally excluded from this project:

\begin{itemize}
  \item \textbf{PushGateway integration:} The library will support pull-based scraping only. Push-based metric transport (e.g., batch job metrics) is not addressed.
  \item \textbf{Advanced metric types:} Specialized types like \texttt{info}, \texttt{enum}, or \texttt{state\_set} will not be implemented in the initial version.
  \item \textbf{Built-in HTTP server:} The client will not expose its own HTTP endpoint; instead, it will provide raw formatted output to be integrated with external frameworks like Ktor.
  \item \textbf{JVM internals instrumentation:} Metrics related to garbage collection, memory pools, or thread activity are considered outside the scope of this project.
\end{itemize}
