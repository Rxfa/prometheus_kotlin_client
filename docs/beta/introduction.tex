%
% Chapter 1
%
\chapter{Introduction} \label{ch:introduction}


\section{Context and Motivation}\label{sec:context-and-motivation}
Over the past few decades, we have been witnessing remarkable advancements on our software systems
and their architecture.\\

The first computers, mainframe computers, Harvard Mark I and the \ac{ENIAC} showed up in the 1930s-1940s and where
developed for military and research purposes.
These were large, powerful hardware machines that took up to an entire room.
The software architecture for these applications was monolithic, meaning that they were simple, self-contained and
independent of other applications~\cite{orkes_software_architecture_evolution, wikipedia_monolithic}.\\

Networks connect and facilitate communication between computers--mainframe to terminal, mainframe to mainframe and
later client to server.
The development of network technology from 1958 onwards enabled mainframes to be connected
electronically, transforming them from isolated machines to multi-user computers that were connected to multiple
terminals.
\ac{ARPANET} was the first public, wide-area computer network, going live in 1969.
It communicated using
packet switching, which went on to serve as the foundation for modern-day Internet as we know it.
Network technology popularized the client-server structure in the 80s, where an application is divided into a
server and a client communicate over a network.
This is a very familiar structure to us today: a client, typically a
desktop computer, remotely makes a request to a server, which returns a response.
This way we can partition tasks
and workloads, the server will work on the processing and retrieval of a piece of data, and the client will in turn
present it~\cite{orkes_software_architecture_evolution, wikipedia_client_server_model}.\\

1983 marked the year of the Internet, a global system of computer networks that used the \ac{TCP/IP} protocol to
facilitate communication between devices and applications.
This was the backbone for \ac{FTP} programs, \ac{SSH} systems, and of course, the World Wide Web.
The invention of the web and its latent possibilities soon kicked off the next wave of application development.
Instead of building a dedicated client for your client for your application, you could simply build a website to be
hosted on the web~\cite{orkes_software_architecture_evolution}.\\

As application development grew, a monolithic codebase became more unwieldy to manage, and it became clear that
capabilities and data stored in a system could be reused.
To address this pain point, modularization became a topic of discussion, and in the 90s the first N-tiered
applications showed up, for the first time we saw the server being split into two tiers: the application server and
the database, the application held all the application and business logic, while the database server, stored the
data recorded, which reduced latency at high processing volumes.
Around the same time, \ac{SOA} emerged as an architectural pattern, and we saw applications split into multiple
services, each design to perform a specific job which could be things like processing payments or verifying user
identities, this had benefits such as an increased scalability as single services could be scaled up or down
depending on demand without having to make changes to the entire system
~\cite{orkes_software_architecture_evolution, oracle_soa, port_soa}.\\

\ac{SOA} set the stage for the move from traditional desktop applications to a new mode of software applications - \ac{SaaS},
but it was the invention of virtual machines and cloud computing that further spurred the explosion of \ac{SaaS}
products in the coming decades.
Machine virtualization exists since the 1960s, but this technology only came into
mainstream use in the 2000s.
It at this period that Amazon, followed by companies like Google, Microsoft and Oracle identified the lucrative that
virtualization offered: managed cloud computing services.
With cloud computing services like Amazon \ac{EC2}, companies could rent \ac{VM}s for processing power and scale as
needed~\cite{orkes_software_architecture_evolution}.\\

The 2010s were the culmination of multiple trends towards distributed computing.
Fueled by the need for third-party access to their services, the first commercial \ac{API}s were lauched in the
2000 by Salesforce and eBay, to enable their costumers and partners to integrate features onto their own sites and
applications.
From Google Maps and Stripe to Twilio and OpenAI, the \ac{API} economy has ballooned since, powering integrated
features across the web.
In the same vein, microservices took off when scaling companies like Netflix and Amazon needed to speed up and
streamline the development cycle, which was slowed by a monolithic architecture.
By splitting up an application into individual microservices, each with its own database, teams could independently update and deploy them, leading to
faster releases and improvements.\cite{orkes_software_architecture_evolution}
The growth in the adoption of this architectural style was heavily supported by the emergence of containers.\\

As we move into the present decade, software systems have become increasingly complex and distributed.
Modern applications often consist of hundreds or thousands of microservices, deployed across hybrid and multi-cloud environments, communicating asynchronously via APIs, message queues, and event streams.
Containers and orchestration platforms like Kubernetes have become the de facto standard for managing these
distributed components at scale\cite{devtron_kubernetes}.
While this architectural evolution brings immense benefits—such as scalability, resilience, and faster development cycles—it also introduces significant operational challenges.
The large number of independent moving parts, dynamic scaling, and network dependencies make it difficult to predict
and diagnose failures or performance degradation using traditional reactive methods.\\

Consequently, couple with more stringent \ac{SLA}s than ever, for multiple reasons, organizations now emphasize
observability, the
ability to
measure a
system's
current
state based
on its
external outputs, recorded as metrics, logs, and traces\cite{dynatrace_observability}.
Observability enables a proactive approach to maintaining a systems reliability and performance standards, allowing
teams to detect
anomalies
early,
understand complex interactions, and respond swiftly before issues impact end users.\\


In this context, metrics-based monitoring, has become essential for maintaining the health of modern distributed systems.
This project aims to fill a notable gap in the Kotlin ecosystem by providing a robust and idiomatic Prometheus client library tailored specifically for Kotlin applications.
Unlike existing libraries that primarily target other JVM languages or lack native support for Kotlin’s coroutine model, this library offers seamless integration and out-of-the-box support for popular Kotlin-only frameworks such as Ktor.
By delivering a solution designed with Kotlin’s language features and concurrency paradigms in mind, this project enables developers to instrument their applications effectively and embrace modern metrics-based monitoring with minimal friction.

\section{Objectives}\label{sec:objectives}

The primary objective of this project is to develop a Kotlin-native Prometheus client library that addresses existing limitations in the Kotlin ecosystem by providing a robust, idiomatic, and efficient solution for metrics instrumentation and collection.

Specifically, the project aims to:

\begin{itemize}
    \item Design and implement a client library that fully leverages Kotlin’s coroutine-based concurrency model to enable asynchronous and efficient metric collection.
    \item Provide seamless integration with popular Kotlin frameworks such as \texttt{Ktor}, ensuring ease of use and
    compatibility within typical Kotlin application environments.
    \item Ensure the library is designed with readability and idiomatic Kotlin style in mind, making it intuitive and enjoyable to use.
    \item Develop an expressive and idiomatic \ac{DSL} for defining and managing metrics,
    reducing boilerplate code and improving developer productivity.
    \item Optimize the library for performance and resource efficiency, surpassing existing Java-based Prometheus
    clients when used in Kotlin applications where possible.
    \item Support \ac{KMP}, enabling broad platform compatibility and facilitating code reuse across \ac{JVM},
    Android, and
    other supported targets.
\end{itemize}

By achieving these objectives, the project intends to fill a critical gap in Kotlin observability tooling and empower developers to instrument their applications with minimal effort while adhering to modern Kotlin programming paradigms.

\section{Structure of the Report}\label{sec:structure-of-the-report}

This report is organized into eight main chapters, each focusing on a specific aspect of the project:

\begin{itemize}
    \item \textbf{Chapter 1: Introduction} — Presents the context and motivation behind the project, outlines the objectives, and provides an overview of the report’s structure.

    \item \textbf{Chapter 2: Background and Motivation} — Covers foundational concepts related to observability and Prometheus, reviews existing Prometheus client libraries, and highlights gaps within the Kotlin ecosystem that motivate this work.

    \item \textbf{Chapter 3: Problem Statement and Goals} — Defines the specific limitations identified in current tooling and states the precise goals and scope of the project.

    \item \textbf{Chapter 4: Design and Architecture} — Describes the high-level architectural decisions, focusing on concurrency models, coroutine usage, and the lifecycle of metric registries.

    \item \textbf{Chapter 5: Implementation Details} — Details the concrete implementation of the client library, including metric types, collector logic, and framework integrations such as with Ktor.

    \item \textbf{Chapter 6: Evaluation and Limitations} — Presents the functional validation, performance evaluation, and comparison with other clients, as well as known limitations of the project.

    \item \textbf{Chapter 7: Future Work} — Suggests possible extensions and improvements, including support for additional technologies, enhanced integration with Kotlin frameworks, and automatic JVM metrics collection.

    \item \textbf{Chapter 8: Conclusion} — Summarizes the contributions made by the project, reflects on lessons learned, and offers final remarks.
\end{itemize}

The report concludes with a list of references and an appendix containing supplementary material.
